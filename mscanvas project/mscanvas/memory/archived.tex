 ::f0.png@
--image begintrig.png im|\textbf{SINUSAS IR KOSINUSAS}
\im{0.4}::f6.png@
\Huge $\sin^2{\alpha}+\cos^2{\alpha}=1$::f7.png@
Pitagoro teorema::f8.png@
\Huge inf ir sup
Tegu $\{a_n\}$ yra seka
$\sup(\{a_n\})=\CC{$\max(\{a_n\}$, jei max egzistuoja}{$\min\Big(\{m: a_n\le m \forall n\}\Big)$, kitais atv.}$
$\inf(\{a_n\})$ - pagal analogiją::f9.png@
Tarkime turime netaisyklingą iškilą figūrą
Tegu $\CC{$\{q\}$ visų įbrėžtinių daugiakampių plotų aibė}{$\{q'\}$ bus visų apibrėžtinių daugiakampių plotų aibė}$::f10.png@
Tegu
$\CC{$Q=sup\{q\}$ yra išorinis plotas}{$Q'=inf\{q'\}$ yra vidinis plotas}$::f11.png@
$Q \le Q'$::f12.png@
$\cc{Q \le \text{ plotas}}{Q' \ge\text{ plotas}}$::f13.png@
Pavyzdys1::f14.png@
55::f15.png@
55::f16.png@
Tegul $\{x_n: n \in \N\}$ yra aibei $\R+\{-\infty, +\infty\}$ priklausančių skaičių seka ir $\cc{y_n=\inf(\{x_n, x_{n+1}, ...\})}{Y_n=\sup(\{x_n, x_{n+1}, ...\})}$. Tuomet $\{y_n\}$ ir $\{Y_n\}$ turi ribas. Jas žymėsime $\displaystyle \lim_{n\to \infty}inf \{x_n\}$ ir  $\displaystyle \lim_{n\to \infty}sup\{x_n\}$::f17.png@
Tegul $\{x_n: n \in \N\}$ yra aibei $\R+\{-\infty, +\infty\}$ 
priklausančių skaičių seka ir $\cc{y_n=\inf(\{x_n, x_{n+1}, ...\})}{Y_n=\sup(\{x_n, x_{n+1}, ...\})}$. Tuomet $\{y_n\}$ ir $\{Y_n\}$ turi ribas. Jas žymėsime $\displaystyle \lim_{n\to \infty}inf \{x_n\}$ ir  $\displaystyle \lim_{n\to \infty}sup\{x_n\}$::f18.png@
Tegul $\{x_n: n \in \N\}$ yra aibei $\R+\{-\infty, +\infty\}$ 
priklausančių skaičių seka ir $\cc{y_n=\inf(\{x_n, x_{n+1}, ...\})}{Y_n=\sup(\{x_n, x_{n+1}, ...\})}$. 
Tuomet $\{y_n\}$ ir $\{Y_n\}$ turi ribas. 
Jas žymėsime $\displaystyle \lim_{n\to \infty}inf \{x_n\}$ ir  $\displaystyle \lim_{n\to \infty}sup\{x_n\}$::f19.png@
Realiųjų skaičių seka $\{x_n: n \in \N\}$ turi ribą tada ir tik tada, kai 
$\displaystyle \lim_{n\to \infty}inf \{x_n\} = \lim_{n\to \infty}sup\{x_n\}$::f20.png@
Tegul $\{f_n(x): n\in \N\}$ yra mačių aibėje $A$ funkcijų seka. Tuomet funkcijos $\displaystyle \inf_n f_n(x)$ ir $\displaystyle \sup_n f_n(x)$, $\displaystyle \lim_{n\to \infty}inf \{f_n(x)\}$ ir \displaystyle \lim_{n \to \infty}sup\{f_n(x)\}$ taip pat yra mačios aibėje $A$.::f21.png@
$\sin A + \sin B = \sqrt{2}$::f22.png@
$(\sin A + \sin B)^2 =2$::f23.png@
Abejų lygybės pusių kėlimas kvadratu::f24.png@
$\sin^2 A + 2\sin A\cos A+\cos^2 B =2$::f25.png@
$(a+b)^2=a^2+2ab+b^2$::f26.png@
$2\sin A\cos A+1=2$::f27.png@
$\cos^2 A+\sin^2 A=1$::f28.png@
burgis::f29.png@
burgis::f30.png@
burgis::f31.png@
burgis::f32.png@
burgis::f33.png@
burgis::f34.png