\documentclass{article}

\usepackage[utf8]{inputenc}
\usepackage[L7x]{fontenc}
\usepackage[lithuanian]{babel}
\usepackage{lmodern}

\usepackage[many]{tcolorbox}
\usepackage{framed}
\usepackage{mdframed}
\usepackage{animate}
\usepackage{amsmath}
\usepackage{tasks}
\usepackage[top=2cm, bottom=2cm, left=1.4cm, right=1.4cm, footskip=1cm, a4paper]{geometry}
\usepackage{hyperref}
\usepackage{subfigure}
\usepackage{tabularx}
\begin{document}
Moksleiviai dažnai neatsimena, kaip spręsti uždavinius, kuriuos jie mokėsi spręsti prieš kelias pamokas, o mokytojai dažnai nustemba, kaip tokių lengvų uždavinių galima neišspręsti. Supaprastintas šio reiškinio paaiškinimas būtų tai, jog matematikos mokymosi sėkmingumas priklauso ne vien nuo ilgalaikio autoritariniu būdu mokomų procedūrų atlikinėjimo tol, kol jos yra įsisavinamos, bet ir nuo įvairių mąstymo būdų, suderintų su besimokančiojo abstraktaus mąstymo sugebėjimais, taikymo mokymosi metu (G. Harel, 2008). Toks paaiškinimas skatina (Simonas, 2018) kelti daugybę klausimų apie abstraktų mąstymą: 
\begin{itemize}
\item Kaip greitai įvertinti moksleivio abstraktaus mąstymo gebėjimus?
\item Kaip atrodo uždaviniai, kuriuose abstraktaus mąstymo gebėjimų reikia labiausiai?
\item Kiek mokantis spręsti konkrečius matematinius uždavinius rezultatus nulemia atkaklus darbas, o kiek abstraktaus mąstymo gebėjimai?
\item Kaip galima abstraktaus mąstymo gebėjimus ugdyti pamokose?
\end{itemize}

Šis tekstas yra tarsi ankstesnio straipsnelio \textit{Vystymosi įtaka matematikos mokymuisi} tęsinys. Straipsnelyje buvo pateiktas modelis, pagal kurį vertinama raida, o čia abstraktus mąstymas negriežtai vadinamas mąstymu, kuris pasižymi formalių operacijų stadijos savybėmis. O šioje dalyje aptarsiu galimus testavimo būdus, paremtus šiuo modeliu ir pasiūlytus įvairiuose pasaulyje atliktuose tyrimuose, lietuvių moksleivių rezultatuose bei pasiūlytus mano paties. 

\section*{Dabartinis darbinis modelis, skirtas įvertinti abstrakčiam mąstymui}
Melvin Thornton savo straipsnyje cituoja J. D. Herron (1975),  L. Copes (1975) ir darbą, kur siūlo būdą gebėjimams įvertinti daug detaliau.\\
\\
\noindent\begin{tabularx}{\textwidth}{|X|X|}
\hline
Konkrečios operacijos & Formalios operacijos \\ \hline 
Atlikti matavimus ir stebėjimus & Įvertinti tokius dydžius kaip greitis ir pagreitis arba plotas ir tūris be tiesioginio matavimo \\ \hline 
Teisingas atsakymas į klausimą \textit{ko paveikslėlyje daugiau: stačiakampių ar kvadratų}
tada, kai suvokiama, kad visi kvadratai yra stačiakampiai & pasirinkti, kuris teiginys teisingas: \phantom{xxxxxxxxxxxxxxx} $\begin{array}{l}\text{Visi stačiakampiai yra kvadratai} \\ \text{Dalis stačiakampių yra kvadratai} \\ \text{Tarp stačiakampių nėra kvadratų} \end{array}$ \\ \hline 
Išrikiuoti lazdelių rinkinį pagal dydį & Mintinai įvertinti, kas aukščiausias, jei Deividas yra aukštesnis už Dovydą, bet mažesnis už Darių \\ \hline 
Atlikti elementarias aritmetines operacijas & taikyti perstatymo ir jungimo savybes skaičiavimuose \\ \hline 
Pertvarkyti algebrinius reiškinius, kartu ir trupmenas & Nustatyti, kaip pasikeis $y$, jei $x$ padidės duotose lygybėse $3x^2$ ir $y=\frac{1}{x}$ \\ \hline 
Apibendrinti gautus duomenis, pvz. nurodyti, kad visų kvadratinių reiškinių $x$ atžvilgiu grafikai atitinka paraboles & Apibendrinti reiškinius nepriklausomai nuo į juos įeinančių kintamųjų, pvz. suvokti, kad paraboles gali atitikti reiškiniai, nebūtinai turintys $x$\\ \hline
\end{tabularx}

Remdamasis savo vėlesnio darbo patirtimi, jis siūlo praplėsti sąrašą (dar neišversta):\\
\\
\noindent\includegraphics[width=0.5\textwidth]{"further_analysis".png}
\section*{Testai, kurių rezultatai žinomi}
Skliausteliuose esantys duomenys reiškia, jog buvo surengta baigusių mokyklą dalyvių apklausa, ir nurodo vietą, datą, dalyvių skaičių ir teisingų atsakymų procentinę dalį.

\begin{enumerate}
\item (1944m., York University, Torontas, <61\%) Įsivaizduokite, jog rankose suspaudžiate molinį rutulį ir pakeičiate jo formą. Nustatykite, ar pasikeis:
\begin{tasks}(3) 
\task Molio kiekis.
\task Rutulio svoris.
\task Rutulio tūris.
\end{tasks}
\item(1973m., University of Oklahoma, 185, 72\%) Abu rutuliai yra vienodo tūrio. Kuris rutulys išstums (iš sklidino indo) daugiau vandens: sunkesnis ar lengvesnis?
\item (1944m., York University, Torontas, <63\%) Žinoma, kad pasibaigus antram pasauliniui karui dalis negailestingų vyrų susilaukė žiaurios mirties. Kadangi Heidrichas buvo vienas negailestingiausių fašistų budelių, tai (\textit{užbaikite sakinį}):
\begin{itemize}
\item Heidrichas, fašistų budelis, susilaukė žiaurios mirties.
\item Heidrichas galbūt susilaukė žaurios mirties.
\item Heidrichas nesusilaukė žiaurios mirties.
\item Nei vienas iš šių teiginių nėra logiškai teisingas.
\end{itemize}
\item Žinoma, kad visi mokytojai yra pedagogai. Įvardykite, kuris teiginys teisingas:
\begin{tasks}(3) 
\task Visi pedagogai yra mokytojai.
\task Kai kurie pedagogai yra mokytojai.
\task Nėra pedagogų, kurie nebūtų ir mokytojai.
\end{tasks}
\item  (1981m., Thornton and Fuller, >1000, 60\% of verbal, 80\% of symbolic). Saulė įstrižai šviečia į penkių metrų aukščio medį ir medis meta septynių metrų šešėlį. Kokio ilgio šešėlį meta kitas medis, kurio aukštis yra trys metrai? [Sąlyga pakoreguota]
\item  (2014m. neprivalomas VBE, 50\%) Automobilio greitis 25 proc. didesnis už motociklo greitį. Apskaičiuokite motociklo greitį, jei automobilio greitis yra 85 km/h.
\item (perfrazavus 2015m. neprivalomame VBE, 74\%) Kubo kraštines padidinome trigubai. Kaip pasikeitė šio kubo tūris?
\newcounter{enumTemp}
\setcounter{enumTemp}{\theenumi}
\end{enumerate}

\section*{Papildomi klausimai, pagal Melvin C. Thornton}

\begin{enumerate}
\setcounter{enumi}{\theenumTemp}
\item Ant stalo padėti du buteliukai: vienas 3 decimetrų aukščio, o kitas 5 decimetrų. Aukštesnio buteliuko šešėlis yra nutįsęs 7 decimetrus. Kiek nutįsęs žemesnio buteliuko šešėlis?
\item Keliais procentais sumažėjo dydis, jei jo reikšmė sumažėjo nuo 8 ligi 6?
\setcounter{enumTemp}{\theenumi}
\end{enumerate}

\section*{Mano interpretacija}

\begin{enumerate}
\setcounter{enumi}{\theenumTemp}
\item Pateikite po pavyzdį, kaip apskaičiuotumėte šiuos dydžius (be matavimo prietaisų): greitį, pagreitį, plotą ir tūrį.
\item Keturi draugai valgė ledus. Mikas suvalgė daugiau nei Tadas, Jonas suvalgė daugiau nei Vytas, Jonas suvalgė mažiau nei Tadas. Išrikiuokite berniukus nuo suvalgiusio daugiausiai iki suvalgiusio mažiausiai.
\item Moneta yra metama 4 kartus. Surašykite visas įmanomas šio eksperimento baigtis (baigties užrašymo pavyzdys: SHHS).
\item Kitą savaitę iš 5 galimų miestų A, B, C, D ir E aš nusprendžiau aplankyti tris. Surašykite visus galimus mano pasirinkimus (pasirinkimo užrašymo pavyzdys: ACE).
\item Restoranas siūlo 4 skirtingų rūšių kavas, kurias pažymėkime skaičiais 1, 2, 3 ir 4. Aš nusprendžiau 4 artimiausias dienas išbandyti po kiekvieną prieš tai nebandytą šio rinkinio kavą, tačiau pirmą dieną 2 ir 4 kavos bandyti nenoriu. Surašykite visas galimas eilės tvarkas, pagal kurias aš galiu šias kavas išbandyti (eilės tvarkos užrašymo pavyzdys: 3241).
\item Duotas reiškinys $\frac{1}{x}$. Nustatykite kaip kinta šio reiškinio reikšmė, kai $x$ reikšmė didėja?
\setcounter{enumTemp}{\theenumi}
\end{enumerate}

\section*{Mano interpretacija: platesni tyrinėjimai}
\begin{enumerate}
\setcounter{enumi}{\theenumTemp}
\item Duomenų rinkinyje yra 2000 skaičių. Programuotojas užrašė programą, kuri kiekvieną iš šių skaičių padidina 10 vienetų. Kaip paleidus jo sukurtą programą pasikeičia šių 2000 skaičių suma?
\item Kam daugiausiai gali būti lygus stačiakampio plotas, jei jo perimetras lygus 8?
\item Apytiksliai įvertinkite skaičių, kurį sudauginę su savimi 12 kartų gausite 2.
\item Su kuriuo N dešimtainis skaičiaus 1/N užrašas turės pabaigą?
\item Kaip pasikeis rezultatas, jei dalmenį sumažinsime 3 kartus?
\item Kiek yra dviženklių skaičių?
\item Kaip keisis rezultatas iš jo daug kartų traukiant šaknį?
\item Ar šaknų iš skaičių sandauga lygi šakniai iš skaičių sandaugos?
\setcounter{enumTemp}{\theenumi}
\end{enumerate}
\end{document}