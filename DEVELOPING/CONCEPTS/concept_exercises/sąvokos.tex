\documentclass[a4paper]{article}
\usepackage[utf8]{inputenc}
\usepackage[L7x]{fontenc}
\usepackage[lithuanian]{babel}
\usepackage{lmodern}
\usepackage{amsmath}
\usepackage[top=2cm, bottom=2cm, left=2cm, right=2cm, footskip=1cm, a4paper]{geometry}
\usepackage{tasks}

\begin{document}
\begin{enumerate}
\item Senovės Graikijoje buvo vartojama tokia Pitagoro teoremos formuluotė: \texttt{Kvadratas ant stačiojo trikampio įžambinės BC yra suma kvadratų ant trikampio statinių BA ir AC}. Pabandykite tą pačią mintį išreikšti taisyklingiau vartodami šiuolaikinę matematinę kalbą.
\item \textit{Elementai} - tai didžiausią įtaką matematikos vystymuisi padaręs vadovėlis, išleistas maždaug 300m. prieš mūsų erą senovės graikų matematiko Euklido. Jame pateikti tokie sąvokų apibrėžimai:
\begin{itemize}
\item \textit{Vienetas yra tai, kas pagal prigimtį egzistuoja kaip vienas daiktas.}
\item \textit{Skaičius yra vienetų daugis.}
\end{itemize}
Palygink, kuo ši skaičiaus samprata skiriasi nuo dabartinės skaičiaus sampratos. Gali remtis angliškoje Vikipedijoje nurodytu skaičiaus apibrėžimu:
\begin{itemize}
\item Skaičius yra matematinis objektas, naudojamas skaičiavime, matavime arba numeravime.
\end{itemize}
\item Įvardykite matmenis šių geometrinių objektų: \texttt{atkarpa, kubas, kvadratas, kampas}.
\item Matmenys, kuriuos įvardijote ankstesniame pratime, antikinėje matematinėje kalboje nebuvo vartojami. Remdamiesi šiuo teiginiu paaiškinkite, kodėl Senovės Graikijos matematikai nenustebtų išgirdę pirmame uždavinyje paminėtą formuluotę.
\item Duotas matematinių objektų rinkinys: \texttt{skaičius}, \texttt{taškas}, \texttt{tiesė}, \texttt{aibė}
Suskirstykite juos į dvi grupes pagal panašumą ir tą panašumą apibūdinkite.
\item Ar egzistuoja
\begin{enumerate}
\item  taškas, kuris nepriklauso tiesei?
\item  taškas, kuris nepriklauso jokiai tiesei?
\item aibė, sudaryta ne iš skaičių?
\item  skaičius, kuris nepriklauso realiųjų skaičių aibei?
\item  atkarpa, kuri nepriklauso kitai atkarpai?
\item  atkarpa, kuri nepriklauso jokiai kitai atkarpai?
\item  aibė, kuri priklauso kitai aibei?
\item  aibė, kuri nepriklauso jokiai kitai aibei?
\item  tiesė, kertanti kitą tiesę daugiau nei viename taške?
\end{enumerate}
\item Duoti skaičiai -0,32 ir -0,18. Užrašykite šių skaičių sumos ir skirtumo sandaugą ir apskaičiuokite sudaryto reiškinio reikšmę.

\item Įrodykite, kad bet kuriems dviems skaičiams $a$ ir $b$ visada galioja teiginys: \texttt{jų sumos kvadrato ir jų skirtumo kvadrato suma yra nemažesnė už jų kvadratų sumos ir bet kurio iš jų kvadrato sumą.}

\item Kiek skaičiaus 2018 užraše yra skaitmenų porų, tokių, kad poros narys, skaičiaus užraše esantis pirmiau likusio poros nario, yra už tą narį didesnis?

\item Kada dviejų skaičių kvadratų santykis nėra lygus jų santykio kvadratui?

\item Koks raidinis reiškinys atitinka sakinį \texttt{skaičių a ir b sumos kvadrato ir šių skaičių kvadratų sumos dalmuo?}

\item  Lentoje ant kvadrato viršūnių yra surašyti tam tikri skaičiai, lygūs $a$, $b$, $c$ ir $d$. Bonifacijus atsitiktinai pasirinko vieną iš sudėties ir daugybos operacijų, o paskui ją atlikęs su kiekviena iš gretimose viršūnėse esančių skaičių porų, lentoje užrašė 4 tos operacijos rezultatus. Tada vėl pasirinko vieną iš sudėties ir daugybos operacijų ir ją atlikęs su šiais 4 rezultatais gavo tam tikrą skaičių. 
\begin{tasks}(1)
\task Užrašykite reiškinius, atitinkančius visas galimas šio skaičiaus išraiškas.
\task Nustatykite visas galimas skaičių $a$, $b$, $c$ ir $d$ sumos reikšmes, jei gautasis skaičius lygus 2018.
\end{tasks}

\item Pateikite būdą, kaip bet kuriam racionaliajam skaičiui sudaryti lygtį, kuriai šis skaičius yra sprendinys.

\item Kaip vadinama aibė skaičių, gautų kiek nori kartų naudojant 1 ir leistinas operacijas, jei:
\begin{tasks}(2)
\task leidžiama tik operacija +; 
\task leidžiamos tik operacijos + ir -; 
\task leidžiamos tik operacijos +, - ir $\times$; 
\task leidžiamos tik operacijos +, -, $\times$ ir : ? 
\end{tasks}
\item Įrodykite, kad tarp bet kurių dviejų skirtingų realiųjų skaičių egzistuoja
\begin{tasks}(2)
\task racionalusis skaičius.
\task iracionalusis skaičiius
\end{tasks}
\end{enumerate}
\end{document}