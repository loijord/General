\documentclass[a4paper]{article}
\usepackage[utf8]{inputenc}
\usepackage[L7x]{fontenc}
\usepackage[lithuanian]{babel}
\usepackage{lmodern}
\usepackage{graphicx}
\usepackage[top=2cm, bottom=2cm, left=1.5cm, right=1.5cm, footskip=1cm, a4paper]{geometry}
\usepackage{indentfirst}
\usepackage{framed}
\usepackage{tikz}
\usepackage{listofitems}
\usepackage{xcolor}
\usepackage{verbatim}
\usepackage[unicode]{hyperref}
\usepackage{amsmath,amsfonts,amssymb,amsthm}
\usepackage{minted}
\usepackage{cancel}
\usepackage{framed}
%\usepackage{mathptmx}
\usepackage{calc}
\usepackage{tasks}
\newcommand{\say}[1]{\textbf{\textit{#1}}}
\newcommand{\goto}[2]{\href{\detokenize{#1}}{\textcolor{blue}{#2}}}
\begin{document}
Anksteniame projekte \textit{\textbf{FRAKTALAI}} viena iš užduočių buvo tokia.
\begin{enumerate}
\setcounter{enumi}{10}
\item \textcolor{green}{\say{IDĖJA KITAM PROJEKTUI.}} \goto{https://www.youtube.com/watch?v=M4vqr3_ROIk}{Šiame video} publiką stulbina žmogus, galintis per minutę sudauginti du penkiaženklius skaičius. Ar galėtumėte apibūdinti bent keletą jo naudojamų metodų? Ar šis veikėjas galėtų sumušti pasaulio rekordą?
\end{enumerate}
Čia eilės tvarka pateiksiu klausimus, apie kuriuos kalbėjome su septintoku Simonu 11-11 dienos pamoką. Juos gvildenant ir išryškėjo poreikis prisiminti šią pernai metais nagrinėtą temą. Visas užsiėmimas truko apie valandą ir 15min.
\section*{Klausimai}
\say{Aš} Kaip tau, Simonai, dabar sekasi mokykloje? Ar parsinešei kokių nors pažymių iš matematikos?

\say{Simonas} Sekėsi visai neblogai, gavau 8. Rašėme iš laipsnių, konkrečiau iš pratimų, kur reikia įvairius skaičius kelti kvadratu arba kubu ir buvo daug veiksmų su minusais.

\say{Aš} O kas sukliudė gauti 10? Ar galėjai pasitikrinti savo klaidas?

\say{Simonas} Atrodo viską mokėjau, bet vis pridarydavau klaidų. Tai minusai išsibarstydavo bedauginant, tai atliekant veiksmus su trupmenomis ne toks skaičius pasirašydavo.

\say{Aš} Na, aštuonis gauti iš matematikos yra jau gerai, bet dar per anksti džiūgauti, nes aštuoni reiškia dviejų balų trūkumą iki 10 ir kas žino, galbūt šis trūkumas kitą kartą išaugs dvigubai, nes tos pačios klaidos gali nulemti vis didesnes spragas ateityje. Kaip manai, ko tau stinga, kad kitą kartą turėtum 10?

\say{Simonas} Manau, jog turėčiau būti labiau susikaupęs, sukaupti dėmesį, nedaryti žioplų klaidų.

\say{Aš} Man matyti, kad lyg ir žinai, ką reikia daryti ir tą nuolat darai. Patikrinkim, ar šis sprendimas veiksmingas. Kada paskutinį kartą gavai 10 iš matematikos?

\say{Simonas} Praeitą pusmetį.

\say{Aš} Na va. Vadinasi nėra taip ir lengva. Mintis, kad užtenka susikoncentruoti, kad pavyktų gauti 10, neatitinka tikrovės. Iš tiesų mąstymo procesai, vykstantys galvoje sprendžiant uždavinius kur kas sudėtingesni. Mes per matematikos pamokas atliekame daugybę procedūrų, t.y. daug ką žinome, kaip daryti žingsnis po žingsnio, bet negalime išvengti klaidų. Taip buvo ir per tavo atsiskaitymą. Klaidas darome pagrinde todėl, kad nemokame tinkamai išnaudoti savo darbinės atminties. Ar esi girdėjęs apie darbinę atmintį?

\say{Simonas} Taip, mums pasakojo.

\say{Aš} Ką apie ją žinai?

\say{Simonas} Ji veikia maždaug kaip RAM'as.

\say{Aš} Taip, visiškai teisingai. Mes galime laikyti esamu momentu tik ribotą kiekį informacijos. Galime panagrinėti tokį pratimą. Užrašau bet kokį skaičių, koks į galvą šauna: 2807967. Esu paklausęs vieno moksleivio, ar jis galėtų įsiminti šį skaičių per 10 sekundžių. Jis atsakė, kad sektųsi sunkiai, tačiau turint tik penkis skaitmenis 28079 būtų lengva. O kaip tau atrodo?

\say{Simonas} Man tai būtų lengva. Tiesa, netgi būtų dar lengviau su aštuoniais skaitmenimis, nes tada aš galiu juos atsiminti iki šimtų.

\say{Aš} Įdomu. Tikrai keista, kad 8 skaitmenys tau paprasčiau. Ką reiškia iki šimtų?

\say{Simonas} Nu pavyzdžiui įsimenu 2807 kaip du skaičius, tada 96 kaip dar vieną, o jei šalia 7 ką nors prirašytume irgi būtų lengva įsiminti.

\say{Aš} Man tai netikėta. O aš įsiminimui panaudočiau aritmetinius ryšius. Pavyzdžiui 28 yra 7 kartotinis, o 96 - šiaip galingas skaičius ta prasme, kad jis turi labai daug daliklių. Man pačiam padeda ne tik išmokti skaičius kaip eilėraštį, bet ,,sutvirtinti'' skaičius aritmetiniais ryšiais, kad jie taip pat tvirtai laikytųsi ir atmintyje. Tačiau grįžkim prie tavo kontrolinių darbų. Ar mokėtum pasinaudoti žiniomis apie darbinę atmintį atliekant matematines procedūras?

\say{Simonas} O šito tai nemokėčiau.

\say{Aš}  Imkime geriausiai tai atitinkantį pavyzdį - daugybą stulpeliu. Ar galėtum atlikti veiksmą $78 \times 23$?

\say{Simonas} $\begin{array}{c}
\phantom{\times0}78\\
\underline{\times\phantom{0}23}\\
\phantom{\times0}234\\
\underline{\phantom\times158\phantom0}\\
\phantom\times1814
\end{array}$

\say{Aš} Tai šitas veiksmas tau nesigavo. Vietoj 158 turėjo būti 156.

\say{Simonas} Ups! Jau antrąsyk mintyse tą pačią klaidą padariau. 

\say{Aš} Teisingas rezultatas tau nesigavo. Nors didžiąją dalį darbo atlikai gerai. Tai matyt reiktų rašyti 7 - 8 balus už pastangas. Vis dėlto šį bei tą, norėdamas palengvinti atminties darbą tu atlikai. Pavyzdžiui užsirašei ant popieriaus skaičius 234 ir 158, kad jų nereiktų atsiminti, kad nereiktų naudoti daug RAM'o. O dabar pabandyk tu man sugalvoti du dviženklius skaičius, kuriuos sudauginčiau.

\say{Simonas} Tegu bus $89 \times 44$.

\say{Aš} (\textit{imu užrašinėti atsakymą}) $\begin{array}{c}
\phantom{\times0}89\\
\underline{\times\phantom{0}44}\\
\phantom{\times044}6
\end{array} \Rightarrow 
\begin{array}{c}
\phantom{\times0}89\\
\underline{\times\phantom{0}44}\\
3\phantom{\times44}6
\end{array}\Rightarrow 
\begin{array}{c}
\phantom{\times0}89\\
\underline{\times\phantom{0}44}\\
3\phantom{\cdot}9\phantom{\cdot}1\phantom{\cdot}6
\end{array}$
 
\say{Simonas} Kaip jums taip greit pavyko?

\say{Aš} O kaip tau, Simonai, atrodo? Juk mes abudu mokame dauginti dviženklius, tai ką aš žinau tokio, kas man leistų gerokai lengviau dauginti dviženklius?

\say{Simonas} Matyt praktika. Daug skaičiavote greičiausiai. 

\say{Aš} Taip, sutinku. Tačiau noriu atskleisti šį tą daugiau apie tai, kokios tai žinios. Iš tiesų, tai aš naudojau keletą būdų, kaip otimizuoti skaičiavimą. Iš pradžių man palengvėjo, kai parašei 44, nes dauginant stulpeliu 89 reikia dauginti tik iš 4 ir daugiau nieko. Aš dauginau pagal tokias lygybes: $$89 \times 4 = (90-1) \times 4 = 360 - 4 = 356$$. 

\say{Simonas} Šitas triukas su atėmimu iš tikrųjų geras, tik aš jį vis pamiršdavau. 

\say{Aš} Taip, ir ne šiaip geras. Jį apsimoka naudoti beveik visada, kai tik kokiame nors skaičiuje iki pilnos dešimties lieka nedaug, pavyzdžiui 1 ar 2 vienetai. Be to, tai didelis RAM'o sutaupymas. Reikėjo padaryti daug paprastesnius dalykus: padauginti $9\times 4$, sumažinti rezultatą vienetu ir prirašyti tai, ką gauname iš 10 atėmę 4. Šitą triuką siūlau prisiminti. 

\say{Simonas} O kodėl jūs iš pradžių rašėte iš galo, po to iš priekio, o po to per vidurį? Taip juk stulpeliu niekas nedaugina.

\say{Aš} Taip, dariau norėdamas neperkrauti darbinės atminties. Aš žinojau, kad stulpelyje reikės dusyk parašyti skaičių 356. Vienas užrašas bus paslinktas į priekį, o kitas atgal. Bet kuriuo atveju vienetų skyrius bus nededamas, todėl rašiau 6. Viduryje reikia atlikti veiksmą 35+56, kas bus mažiau už 100, todėl tūkstančių skyrius taip pat nededamas. Tai suvokęs įrašiau priekyje 3, o paskui perėjau prie to veiksmo. Šiek tiek ėmė painiotis tie penketai, norėjau rašyti 90, bet po to apsigalvojau ir parašiau 91. Daugiausiai RAMo sunaudojo veiksmo 35+56 atlikimas atmintyje. Čia aš ėmiau pasiekinėti savo RAM'o ribas - man truputį ėmė painiotis. Vadinasi, mano darbinė atmintis sugeba tik atlikti veiksmą 35+56 ir gal dar šį tą nedidelio.

\say{Simonas} Oho, daug gudrybių. Galbūt dar kažką galima kažko sugalvoti?

\say{Aš} Taip, pasakysiu vieną labai įdomų triuką. Tokį su kvadratais. Tik gaila, kad pagal mokyklinę programą jį pirmą kartą galima pamatyti tik aštuntoje klasėje. Imkime tarkime skaičius 78 ir 44 ir sudauginkime. Rašau $78 \times 44 =$, bet iš karto nedauginsiu, o vietoj to paklausiu, koks šių abiejų skaičių vidurkis ir kiek jie abudu nuo jo skiriasi. Ar galėtum rasti vidurkį?

\say{Simonas} Kieno? 7844?

\say{Aš} Skaičių 78 ir 44.

\say{Simonas} Sunku.

\say{Aš} Bet juk savo pažymių vidurkį tai moki apskaičiuoti?

\say{Simonas} Palauk, pabandysiu. Nu man atrodo, kad spėliočiau. Gal imti kažką per vidurį, pavyzdžiui 59, tada žiūrėti ar tinka.

\say{Aš} Taip tai tikrai niekas vidurkių neskaičiuoja. Aš pats tai bandyčiau imti tų skaičių skirtumą ir dalinti jį pusiau...

\say{Simonas} Ai, prisiminiau. Kai sudedu, gaunu 122, tada padalinu iš 2 ir gaunu 61.

\say{Aš} Na va pagaliau. O dabar ką gautum, jei atimtum 61 iš abiejų skaičių?

\say{Simonas} Gaučiau 17 ir -17.

\say{Aš} Taip, viskas teisinga, dabar padarysim šiek tiek magijos. Žiūrėk: $78 \times 44 = 61^2 - 11^2$

\say{Simonas} Palauk, o kaip tai įmanoma?

\say{Aš} Tai sakau, kad tau bus naujiena. Tu kol kas neturėtum būt matęs pagal mokyklinę programą, kodėl tai galioja. Kaip formulę šią taisyklę galima užrašyti šitaip: $$(a-b)(a+b) = a^2 - b^2$$ Iš tikrųjų, kairę pusę galiu paaiškinti taip: yra dauginami du skaičiai, vienas skiriasi nuo vidurkio į vieną pusę, kitas į kitą. Tai 78 skiriasi per 17 nuo 61, o 44 - per 17 į kitą pusę. Pagal taisyklę tada sandauga lygi 61 ir 17 kvadratų skirtumui.

\say{Simonas} Aš nesuprantu. Iš kur tie kvadratai?

\say{Aš} Nieko baisaus, juk jūs šito dar nėjot. Ir net nesat matę loginių pagrindų, pagal kuriuos šita taisyklė galioja. Nors... Pernai metais mes kartu visus tuos pagrindus išmokom. Tik prie dviženklių dauginimo nepriėmo.

\say{Simonas} Aš nieko neatsimenu.

\say{Aš} Ar atsimeni, kaip ėjom apie fraktalus?

\say{Simonas} Šitą atsimenu. 

\say{Aš} Tai va, ten daugindavo kažką lentelių pagalba. Pasirodo, kad fraktalus paaiškina tas pats dalykas, kaip ir dviženklių skaičių daugybą. Gal dabar jau prisiminei taisyklę $(a-b)(a+b) = a^2 - b^2$?

\say{Simonas} Tikrai neatsimenu. 











\end{document}